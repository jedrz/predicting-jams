\documentclass[a4paper,12pt]{mwart}

\usepackage{polski}
\usepackage[utf8]{inputenc}
\usepackage{float}
\usepackage{color}
\usepackage{hyperref}

\newcommand{\TODO}[1]{\textcolor{blue}{TODO: #1 \\}}
\newcommand{\ang}[1]{ang.~{\itshape #1}}

% http://tex.stackexchange.com/questions/83440/inputenc-error-unicode-char-u8-not-set-up-for-use-with-latex
\DeclareUnicodeCharacter{00A0}{~}

\begin{document}

\title{Predicting Jams\\%
{\large czyli model prognozujący tworzenie się korków samochodowych} }

\author{Łukasz Jędrzejewski \and Artur Sawicki}

\maketitle

\section{Opis projektu}
W ramach projektu realizowaliśmy zadanie polegające na zbudowaniu modelu prognozującego tworzenie się korków na podstawie historychnych obserwacji. Zadanie opiera się na zadaniu ze strony \href{http://tunedit.org/challenge/IEEE-ICDM-2010/jams}{tunedit.org}.

Do realizacji celów zadania użyliśmy bazy danych \href{https://www.postgresql.org/}{PostgreSQL} rozszerzonej o dodatek umożliwiający pracę z danymi przestrzennymi \href{http://postgis.net/}{PostGIS}. Aby ułatwić pracę na różnych maszynach, baza danych wraz z rozszerzeniem zainstalowana jest na kontenerze typu \href{https://www.docker.com/}{docker}. Skrypty odpowiadające za stworzenie schematu w bazie danych, przetwarzanie i wizualizację danych oraz właściwe obliczenia napisaliśmy w języku \href{https://www.python.org/}{python}.

\section{Opis problemu}
\subsection{Wprowadzenie}
Stacje radiowe wpadły na pomysł zbierania informacji dotyczących zatłoczenia ulic oraz przekazywanie ich do kierowców, w celu umożliwienia im omijania nieprzejezdnych dróg. Takie dane można użyć też w inny sposób - na ich podstawie można przewidzieć, gdzie mogą pojawić się kolejne korki, bazując na początkowym stopniu zatłoczenia ulic. Tego właśnie dotyczy zadanie.

\subsection{Dane}
Dane użyte w zadaniu wygenerowane zostały z jedno-godzinnych symulacji. Każda z nich zaczynała się prawie pustą siecią dróg, do której dodawane są samochody, z losowo wybranym punktem startu i docelowym.

Na początku każdej symulacji 5 losowych odcinków dróg (2 główne i 3 mniejsze) zostawało usuniętych z grafu imitując roboty drogowe. Jako dane wejściowe algorytm dostaje identyfikatory 5 usuniętych dróg oraz sekwencję identyfikatorów dróg, na których korek pojawił się w ciągu pierwszych 20 minut symulacji. Zadaniem jest przewidzenie korków w następnych 40 minutach.

Odcinek uznajemy za zakorkowany kiedy średnia prędkość w ciągu ostatnich 6 minut nie przekroczyła 5km/h, a liczba samochodów jaka przejchała lub znajduje się na nim jest większa niż 10.

Zbiór treningowy i zbiór testowy zawierają po 5000 próbek, z których każda jest pojedynczą symulacją. Dane treningowe zawierają pierwsze 20 minut symulacji oraz kolejne 40 minut, natomiast dane testowe - tylko pierwsze 20 minut.

Dostępny jest także graf ulic.

\scriptsize
\begin{verbatim}
docker cośtam
\end{verbatim}
\normalsize

\section{Podsumowanie}
Ale fajno...

\end{document}
